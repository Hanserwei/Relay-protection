\chapter{牵引降压混合变电所保护配置}

\section{35KV进线保护}
35KV进线选择采用过电流保护,零序过电流保护,以及差动保护,来进行保护配置。
\subsection{过电流保护}
按躲过最大负荷电流整定,满足上下级配合。整定计算公式子如下:
$$
I_{opk}=\frac{K_{rel}\times I_{fh}}{K_r\times n_{TA}}
\\
K_{sen}=\frac{I_{2k2.\min}}{I_{op}}
$$
$I_{2k2.min}$最小运行方式下,被保护线路末端两相短路最小短路电流。\par 
时间:T逐级配合\par 
动作:跳闸
\subsection{零序过电流保护}
按与下级配合系数整定,因过电流保护与零序过电流采用同一整定时限,故可采用系数整定,以提高保护的灵敏度。按躲过外部三相最大短路电流在电流互感器产生的误差整定。
$$
I_{opk}=0.1I_{3k3.\max}
\\
K_{sen}=\frac{I_{d0}}{I_{op}}
$$
时间:T逐级配合\newline 
动作:跳闸
\subsection{差动保护}
按躲过电缆线路稳态对地电容电流,
$$
I_{diff}>=\frac{2\times I_C}{n_{TA}}
$$
$$
I_C=0.1U_EL,\text{并大于}15\%\text{正常进行的额定电流,整定值还需要按照下式考虑,}
$$
$$
I_{diff}>=1.2\times 0.1\times \left( I_{fh}+I_k \right) 
$$
$$
I_{fh}-\text{本供电分区范围除去最大容量变压器的负荷电流}
$$
$$
I_k—\text{本供电分区范围最大容量变压器的励磁涌流};\text{采用保护装置给定的最小值}
$$
时间:0秒\newline
动作:跳闸
\section{35KV出线保护}
对于35KV出线保护配置,也配有过电流,零序过电流,差动保护。保护配置整定计算同35KV进线保护。
\section{35KV母联保护}
\subsection{延时速断保护}
延时速断保护按躲过任一母线段的最大负荷整定,也可按进线电流整定,但时间小于进线值。与整流变,配电变的馈线电流速断和出线的过电流保护的时限配合,均按800A,全线统一。\par 
时间:0.6秒\newline
动作:跳闸
\subsection{零序过电流保护}
零序过电流保护均按200A整定,全线统一。
\subsection{后加速过电流保护}
后加速过电流保护均按800A,全线统一。\par 
整定值:4A\par
整定时间:t=0.3s\par 
动作:跳闸
\subsection{后加速零序过电流保护}
后加速零序过电流保护与后加速过电流保护相同,均按800A,全线统一。\par 
整定时间:t=0.3s\par
作用:跳闸

\section{整流机组保护}
整流机组配有的保护包括:电流速断保护,过电流保护,反时限过电流保护,重负荷,过负荷以及温度保护。
\subsection{电流速断保护}
对于电流速断保护,整定如下:
$$
I_{op.k}=\frac{8\times I_{fh}}{n_{TA}}
\\
K_{sen}=\frac{I_{1k2.\min}}{I_{op}}
$$
时间:0秒\newline 
动作:跳闸
\subsection{过电流保护}
按躲过300\%过负荷能力整定。
$$
I_{op.k}=\frac{4\times I_{fh}}{n_{TA}}
\\
K_{sen}=\frac{I_{1k2.\min}}{I_{op}}
$$
时间:300ms\newline
动作:跳闸
\subsection{反时限过电流保护}
反时限过电流整定原则,应躲过牵引整流机组的V1级负荷特性要求,涵盖重、轻过负荷曲线,选择IEC Very类型。
$$
I_{op.k}=\frac{K_{rel}\times I_{fh}}{n_{TA}}
$$
根据牵引变压器过载曲线求出时间系数时间系数定值,
$$
t=\frac{80}{(\frac{I}{I_P})^2-1}T_p(s)
$$
\subsection{零序过电流保护}
整定计算如下:
$$
I_{op.k}=\frac{K_{rel}I_{fh}\times 0.5}{K_r\times n_{TA}}
$$
时间:0秒 \newline
动作:跳闸
\subsection{重负荷保护}
整定计算如下:
$$
I_{op.k}=\frac{3\times I_{fh}}{n_{TA}}
$$
时间:T过负荷曲线300\%,一分钟。
\subsection{过负荷保护}
整定计算如下:
$$
I_{op.k}=\frac{1.5\times K_{rel}\times I_{fh}}{n_{TA}}
$$
时间:负荷曲线150\%,两小时。
\subsection{温度保护}
整流变压器,相关部件温度超过整定温度,进线报警。
\section{配电变压器保护}
降压所配电变压器保护整定方法与跟随所完全一致,因此对于具体的整定方案此处不多赘述,直接进行整定结果的讨论。
\section{电流速断保护}
时间:0秒\par 
动作:跳闸
\subsection{反时限过电流保护}
动作:跳闸
\subsection{零序过电流保护}
时间:0.25秒\par
动作:跳闸
\subsection{温度保护}
超温报警温度:140摄氏度\par 
超温跳闸温度:150摄氏度

\section{直流1500V系统保护}
\subsection{1500V馈线保护}
装置包括:DPU96、UR36/UR40、DCR150。其中DPU96用在$I_{max}$最大电流、$\Delta I$电流增量、$\frac{di}{dt}$电流上升率、$I_{DMT}$定时限过流、热过负荷保护。UR36/UR40用于大电流脱扣。DCR150用于双边联跳。
\subsubsection{$I_{max}$最大电流}
类似速断保护,小于大电流脱扣,大于电流增量$\Delta I$值。\par 
时间:1ms。\par 
返回系数80\%
\subsubsection{$\Delta I$电流增量}
大于机车启动电流和接触网过分段的冲击电流。\par 
整定值:4000A\par 
$\Delta I$报警功能(如果不注明,则是80\%报警):60\%报警\par 
时间:延时1ms\par 
$\Delta I\frac{di}{dt}$是基于$\Delta I$,为了防止电流增量($\Delta I$)保护受于干扰误动,还附加了延时时间、$\Delta I\frac{di}{dt}$持续时间等条件。需具备3个条件,祥见整定原则。\par 
$\Delta I\frac{di}{dt}$电流上升率返回延时:1ms
\subsubsection{$\frac{di}{dt}$电流上升率}
大于机车启动和接触网过分段冲击产生最大电流变化率,小于线路末端短路时的电流变化率,相当于有两阶段保护。 \par
$\frac{di}{dt}$整定值:40\par 
$\frac{di}{dt}$报警:60\%报警\par 
时间:延时40ms
\subsubsection{$I_{DMT}$定时限过流}类似过电流保护,小于线路末端接触网与接地线发生的最小短路电流,大于车辆启动时的电流和时间,车辆段和末端牵引变电所需适当加大,4000A。\par  
整定值:3000A
\subsubsection{热过负荷保护}
考虑牵引曲线电流上升时间。\par  
整定值:20s\par 
投入:挡位2\par 
牵引网类型:架空接触网\par 
报警系数:0.9\par 
根据电缆材质定:报警温度:65℃\par 
根据电缆材质定:跳闸温度:75℃\par 
返回温度:30℃
\subsection{大电流脱扣}
为断路器本体自带保护,通过断路器内设置的脱扣器实现跳闸。大于最大负荷下列车正常启动电流,小于近端最大短路电流。主要是近端短路的保护,与机车保护配合,躲过机车断路器的大电流脱扣。\par
整定值:8000$\sim$10000A
\subsubsection{双边联跳}
牵引网双边联跳是作为越区保护的一种重要手段 ,指当直流馈线开关保护装置检测到本回路馈线出线大电流故障脱扣后,通过保护装置的硬接点向邻所对应直流馈线开关发出跳信号,联跳对应邻所馈线直流开关。\par  
分为双边联跳与大双边联跳 。\par 
大双边联跳:如果有一牵引所因故障退出运行,其通过纵向隔离开关进行越区供电,如果出现大双边供电模式,联跳也要转换至大双边联跳模式。馈线保护的设置:宁可误动作,不可不动作

\subsection{钢轨电位限制}
钢轨电位限制装置依据人体耐受电压时间特性曲线(E50122-1:12.97),按3级保护配合,并与电压型框架保护整定值配合。电压超过90V时报警;电压达到120V,通过接触器无延时永久短路;电压达到240V(定值可调),动作时限为0秒,通过晶闸管快速接地,接触器闭合后斯开,同时接触器闭锁。
\subsection{负极柜}
\subsubsection{电压型框架保护}
电压型框架保护定值和时间要与钢轨电位限制装置相互配合,为确保台上乘客和工作人员的人身安全。电压型框架保护的电压和时间的整定值尺于钢轨电位限制装置动作。\par 
整定值:90V\par
时间:0.3s\par
动作:报警\par
整定值:130V\par
动作:跳闸
\subsubsection{电流型框架保护}
电流元件灵敏度高,可调整定值,分流1000A/150mV、定值为35A$\sim$85A,动作时限为0秒。\par   
整定值:80A
\addtocounter{page}{-1}
