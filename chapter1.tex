\chapter{继电保护概述}

\section{继电保护任务}
为了实现继电保护装置的任务,必须在技术上满足四个基本要求:选择性、速动性、灵敏性和可靠性。对于用于继电器跳闸的继电保护,它应同时满足这四个基本要求。而对于用于信号传递以及仅反映不正常运行情况的继电保护装置,可以放宽其中一些要求。
\section{继电保护整定原则}
\subsection{继电保护种类及整定原则}
1. 差动电流速断保护\par
整定原则包括躲开设备启动时最大负荷电流、外部短路时的不平衡电流、变压器最大励磁涌流等条件。\par
2. 纵差保护\par
整定原则基于躲过设备启动时的不平衡电流,通常使用比率制动系数K来计算最大制动系数。\par
3. 瞬时电流速断保护\par
整定依据躲过线路末端的最大故障电流。\par
4. 定时限电流速断保护\par
整定需要配合相邻元件末端的最大三相短路电流或相邻元件电流速断保护的动作电流,选择两个条件中较大的整定值。\par
5. 过电流保护\par
整定基于分支线上设备的最大起动电流之和。\par
6. 过负荷保护\par
整定依据额定负荷电流。\par
7. 低电压保护\par
整定需要考虑设备起动时供电母线的最小允许电压,以及可靠系数和电压继电器的返回系数。\par
8. 过热保护\par
过热保护的整定需要考虑发热时间常数和散热时间。\par
9. 接地保护\par
整定基于外部最小单相接地故障电流。\par
\subsection{变压器保护整定原则}
1. 差动电流速断保护\par
整定原则包括躲开变压器的最大负荷电流、外部短路时的最大不平衡电流、变压器最大励磁涌流。\par
2. 零序差动保护\par
整定需要考虑外部单相接地短路时的不平衡电流,低压侧母线三相短路电流,分支线上需要自起动的电动机的最大起动电流之和,以及低压侧零序过电流保护的整定。\par
3. 高压侧过负荷保护\par
对称过负荷保护的动作电流按照额定电流进行整定。\par
\subsection{三段式电流保护}
三段式电流保护包括瞬时电流速断保护、限时电流速断保护和定时限过电流保护。
瞬时电流速断保护作为线路首端的主保护,限时电流速断保护作为近后备和末端的主保护,定时限过电流保护作为近后备和远后备。\par
这种三段式电流保护适用于不同情况,以确保线路的可靠保护。
\subsection{阶段式电流保护}
阶段式电流保护包括无时限电流速断、带时限电流速断和过电流保护。
这些保护装置的组合构成了一整套输电线路阶段式电流保护系统,可以根据具体需要选择装置的数量和类型。

\section{继电保护装置基本要求}
\subsection{选择性}
选择性意味着在电力系统的设备或线路发生短路时,继电保护只需将故障的设备或线路从电力系统中切除。如果故障设备或线路的保护或断路器无法动作,相邻设备或线路的保护应负责切除故障。
\subsection{速动性}
速动性表示继电保护装置应能够尽快切除故障,以减少设备和用户在高电流、低电压条件下的运行时间,从而降低设备的损害程度,并提高系统的并行运行稳定性。
通常需要快速切除的故障包括:
(1) 使发电厂或重要用户的母线电压降至有效值以下(通常为额定电压的0.7倍)。
(2) 大容量发电机、变压器和电动机的内部故障。
(3) 中、低压线路导线截面太小,不允许延迟切除以避免过热。
(4) 可能危及人身安全或对通信系统造成严重干扰的故障。
故障切除时间包括保护装置和断路器的动作时间。通常,快速保护的动作时间为0.04秒至0.08秒,最快可达0.01秒至0.04秒。一般断路器的跳闸时间为0.06秒至0.15秒,最快可达0.02秒至0.06秒。
对于反应非正常运行情况的继电保护装置,通常不需要快速动作,而应根据选择性的条件带有延迟的信号。
\subsection{灵敏性}
灵敏性表示电气设备或线路在受到保护的范围内发生短路故障或不正常运行情况时,保护装置的响应能力。具备良好灵敏性的继电保护能在规定的范围内对各种短路位置和类型作出正确响应,即使存在过渡电阻或在系统的最大和最小运行方式下也能可靠动作。
系统的最大运行方式指被保护线路末端短路时,系统等效阻抗最小,因此保护装置需要能够在最大负荷下快速响应。系统的最小运行方式下,即在同样的短路情况下,系统等效阻抗最大,要求保护装置在此情况下也能可靠动作。保护装置的灵敏性通过灵敏系数来衡量。
\subsection{可靠性}
可靠性包括安全性和可信赖性,这是继电保护的最根本要求。安全性要求继电保护在不需要响应时可靠地保持不响应,以避免误动。可信赖性要求继电保护在规定的保护范围内出现应该响应的故障时可靠地响应,以避免拒动。继电保护的误动和拒动都可能对电力系统造成严重危害。
尽管针对相同的电力元件,随着电网的发展,保护不误动和不拒动对系统的影响也会有所变化。这四个基本要求是设计、配置和维护继电保护的基础,也是分析和评估继电保护性能的依据。虽然这些要求之间可能存在矛盾,但在实际工作中,需要根据电网结构和用户需求进行综合考虑。

\section{继电保护的分类}
1、按被保护对象分类\par
1.	输电线保护
2.	主设备保护(如发电机、变压器、母线、电抗器、电容器等保护)\par
2、按保护功能分类\par
1.	短路故障保护
2.	主保护
3.	后备保护
4.	辅助保护
5.	异常运行保护
6.	过负荷保护
7.	失磁保护
8.	失步保护
9.	低频保护
10.	非全相运行保护等\par
3、按保护装置进行比较和运算处理的信号量分类\par
1.	模拟式保护
2.	机电型
3.	整流型
4.	晶体管型
5.	集成电路型(运算放大器)
6.	数字式保护
7.	使用微处理机和微型计算机的保护装置,反映的是经过采样和模/数转换后的离散数字量。\par
4、按保护动作原理分类\par
1.	过电流保护
2.	低电压保护
3.	过电压保护
4.	功率方向保护
5.	距离保护
6.	差动保护
7.	纵联保护
8.	瓦斯保护等\par
这些分类方式有助于对继电保护装置和系统进行更系统化的管理和理解,以确保电力系统的可靠性和安全性。

\section{继电保护装置维护的相关规程}
一、二次回路检验(通用部分)\par
在进行被保护设备的断路器、电流互感器以及电压回路与其他单元设备的回路完全断开之后,方可进行检验。\par
二、电流互感器二次回路检查\par
检查电流互感器二次绕组的所有二次接线是否正确,端子排引线螺钉是否可靠。
检查电流二次回路的接地点和接地状态。每个电流互感器的二次回路必须只有一个接地点。对于由多组电流互感器二次回路组成的情况,应在有直接电气连接的地方设置一个接地点。\par
三、电压互感器二次回路检查\par
检查电压互感器二次绕组的所有二次回路接线是否正确,端子排引线螺钉是否可靠。
为确保接地的可靠性,不得将各电压互感器的中性线连接到可能断开的断路器或接触器上。对于独立的二次回路,与其他互感器二次回路没有直接电气联系的情况下,可以在控制室或开关站实现一点接地。
检查电压互感器二次回路中所有熔断器(自动开关)的装设地点、熔断(脱扣)电流是否适当(自动开关的脱扣电流需通过试验确定)、质量是否良好,能否保证选择性,以及自动开关线圈阻抗值是否合适。
检查串联在电压回路中的断路器、隔离开关以及切换设备触点接触的可靠性。
测量电压回路自互感器引出端子到配电屏电压母线的每相直流电阻,并计算电压互感器在额定容量下的压降,其值不应超过额定电压的3 \% 。\par
四、二次回路绝缘检查\par
在对二次回路进行绝缘检查之前,必须确认所有断路器、电流互感器已断电,交流电压回路已与其他单元设备的回路断开,并且与其他回路隔离良好。只有在这种情况下才允许进行绝缘测试。
在进行绝缘测试时,需要注意以下事项:\newline
1.	试验线连接必须牢固。\\
2.	在进行每项绝缘试验后,必须对试验回路进行放电。\\
3.	对于母线差动保护、断路器失灵保护和安全自动装置等情况,如果不可能同时停电所有被保护的设备,那么绝缘电阻检验必须分段进行,即每次只测定一个被保护单元所属回路的绝缘电阻。\par
五、断路器、隔离开关二次回路的检验\par
继电保护检验人员需要了解以下信息:
设备的技术性能和调试结果。
电保护屏柜引出到断路器(包括隔离开关)二次回路端子排的电缆线连接的正确性以及螺钉的压接可靠性。
此外,继电保护检验人员还需要了解以下内容:\newline
1.  断路器跳闸线圈和合闸线圈的电气连接方式,包括防止断路器跳跃回路和三相不一致回路等措施。\\
2.	与保护回路相关的辅助触点的状态、开闭时间、构成方式以及触点容量。\\
3.	断路器二次操作回路中气压、液压等监视回路的工作方式。\\
4.	断路器二次回路的接线图。\\
5.	断路器跳闸和合闸线圈的电阻值以及在额定电压下的跳闸和合闸电流。\\
6.	断路器的跳闸电压和合闸电压,其值应符合相关规程的规定。\\
7.	断路器的跳闸时间、合闸时间以及三相触头不同时开闭的最大时间差,不应超过规定值。\par
六、运维和能力提升\par
此部分涉及到对其他设备的运维和技能提升,需要继电保护检验人员了解和熟悉设备的操作和维护程序,以提高设备的可靠性和性能。
\addtocounter{page}{-1}

