\chapter{主变电所保护配置}


\section{110KV侧环进线保护}
进线的主保护,快速切除进线点至主变压器之间的相间短路故障。
\subsection{过电流保护I与Ⅱ---跳闸}
主变压器过电流保护按躲过最大负荷电流整定,要求电流和时间与相邻线路配合,并满足灵敏系数要求,其整定公式如下:
$$
I_{k.act}=\frac{K_{rel}K_{gh}I_{fh}}{K_rn_{TA}}
$$ 
$$
K_{sen}=\frac{I_{2k2.min}}{I_{k.act}}
$$
上式中,$I_{2k2.min}$为最小运行方式下,低压侧发生两相短路时,流过高压侧保护装置安装处的稳态电流。\par 
整定时间:t=2.5s\par 
作用:跳闸

\subsection{零序过电流保护I与Ⅱ---跳闸}
主变压器零序过电流保护按躲过外部三相短路最大不平衡电流整定,要求满足相邻线路故障灵敏系数要求,其整定公式如下:
$$
I_{k.act}=\frac{K_{rel}\varDelta fI_{3k3.max}}{K_rn_{TA}}
$$
上式中,$I_{3k3.min}$为最大运行方式下,低压侧发生三相短路时,流过高压侧保护装置安装处的稳态电流。\par 
整定时间:t=0.3s\par
作用:跳闸
\section{110KV侧环出线保护}
预留环出扩展条件,将来环出回路上时,保护同步实施,环进、出线需均设线路纵差保护。
变压器110KV侧保护
差动保护---跳闸
\subsection{过电流保护(或低压过流)---跳闸}
零序过电流保护---跳闸,
重瓦斯保护---跳闸,
有载调压重瓦斯保护---跳闸,
过负荷、轻瓦斯、有载调压轻瓦斯、油位异常、气压释放等保护动作发信号
\section{变压器35KV侧保护}
差动保护---跳闸
(设独立的CT线圈与接地变CT线圈---并向高压侧差接。)
主变压器差动保护的最小动作电流按躲过变压器额定负载时的最大不平衡电流整定,整定值为TAPn的倍数,一般整定为0.3$ \sim $ 0.5(TAPn);\par 
双折线比率:按照相关规程及SEL保护配置说明书,取双折线动作斜率SLP1=50,SLP2=80\%;\par 
制动电流:因为SLP1折线经过原点,所以该值无需整定,可通过差动保护最小动作电流和折线斜率来间接进行整定,即IRS0=1。ISR1=4主要为了防止区外穿越性故障大电流引起的两侧CT误差增大造成保护误动。\par 
二次谐波闭锁百分比(PCT2):在空投变压器时会在变压器电源测产生很大的激磁涌流,可能引起差动保护误动。激磁涌流在第一周波产生的二次谐波电流与基波电流之比通常超过30\%,因此利用二次谐波电流来识别激磁涌流现象并防止差动保护误动,一般可整定为15\%$ \sim $ 20\%。\par 
作用:跳闸
\subsection{过电流(低压过流)保护---跳闸}
35kV侧公用母线进线的低电压过电流保护按躲过最大负荷电流整定,要求电流和时间与相邻线路配合,并满足灵敏度系数要求,其整定公式如下
$$
I_{k.act}=\frac{K_{rel}K_{gh}I_{fh}}{K_rn_{TA}}
$$
$$
K_{sen}=\frac{I_{1k2.min}}{I_{k.act}}
$$
考虑当远端发生故障时不启动保护,一般将动作电压设为0.15~0.3倍的额定电压大小,电压整定公式如下:
$$
U_{act}=\frac{U_{min}}{K_{rel}n_{TV}}
$$
整定时间:与下级低电压过电流保护配合\newline
作用:低电压时跳闸

\subsection{零序过电流保护---跳闸}
35kV侧公用母线进线的零序电流保护按躲过外部三相短路最大不平衡电流整定,其公式如下:
$$
I_{k.act}=\frac{K_{rel}\varDelta fI_{3k3.max}}{K_rn_{TA}}
$$
若与下级受总零流配合,其计算式为:
$$
I_{k.act}=\frac{K_pI_{act1}}{n_{TA}}
$$
整定时间: t=2.3+Δt \newline
作用:跳闸

\section{35KV接地变保护}
\subsection{电流速断---跳闸}
若接地变压器的速断保护按躲过接地变励磁涌流整定,则取整定值为7~10倍的额定电流,其整定公式为:
$$
I_{k.act}=\frac{9I_N}{n_{TA}}
$$
若接地变压器的速断保护按躲过低压侧单相接地短路整定,则整定公式为:
$$
=\frac{K_k\frac{I_{jd}^{(1)}}{3}}{n_{TA}}
$$
若接地变压器的速断保护按灵敏系数整定,则整定公式为:
$$
I_{k.act}=\frac{I_{1k2.min}}{K_{sen}n_{TA}}
$$
上式中,$I_{1k2.min}$为接地变压器电源侧最小两相短路电流。\newline
作用:联跳主变两侧开关

\subsection{过电流-跳闸}
若接地变压器的过电流保护按躲过接地变压器的额定电流整定,则整定公式为:
$$
I_{k.act}=\frac{K_{rel}K_{gh}I_{fh}}{K_rn_{TA}}
$$
若接地变压器的过电流保护按躲过外部单相接地时流过接地变的最大故障电流整定,则整定公式为:
$$
I_{k.act}=\frac{K_k\frac{I_{jd}^{(1)}}{3}}{n_{TA}}
$$
作用:联跳主变两侧开关
\section{35KV所用变保护}
\subsection{过电流保护}
按躲过变压器0.4kV侧出口三相短路流过高压侧的稳态电流整定。
$$
X^*=\frac{U_k\%\times S_j}{100\times S_n}
$$
$$
I^{*}_{d3}=\frac{1}{X_B}
$$
0.4kV侧三相短路电流
$$
I_{0.4d_{3}}=I^{*}_{d3}\times I_{j0.4}
$$
折算到高压侧:
$$
I_{1k3\cdot max}=\frac{I_{0.4k3max}}{T_{RA}}
$$
$$
I_{opk}=\frac{K_nI_{1k3\max}}{n_{TA}}
$$
时间:t=0秒
\subsection{零序过电流保护}
计算公式如下:
$$
I_{opk}=\frac{0.5I}{n_{TA}}
$$
时间:t=0.5秒
\subsection{反时限过电流保护}
反时限过电流保护按躲过变压器的最大负荷电流整定,整定公式为:
$$
I_{k.act}=\frac{K_{jb}I_{fh}}{n_{TA}}
$$
整定值:0.1A \newline
整定时间:t=0.5s \newline
作用:跳闸

\addtocounter{page}{-1}


